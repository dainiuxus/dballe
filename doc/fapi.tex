%
% Editorial notes are found in the 'Introduction' chapter, marked with a
% comment to find them more easily
%

%\documentclass[draft,a4paper]{book}
%\documentclass[final,a4wide]{book}
%\documentclass[final,12pt,a4paper,twoside]{book}
\documentclass[final,12pt,a4paper,twoside]{book}
%\usepackage[italian]{babel}
\usepackage[ascii]{inputenc}
\usepackage[T1]{fontenc}
\usepackage{graphicx}
\usepackage{moreverb}
\usepackage{ifthen}
% Include alcuni simboli non piú standard ma usati da gnuplot
\usepackage{subfigure}
\usepackage{epsfig}
\usepackage{longtable}

\title{DB-ALLe Guide of the Fortran API}
\author{Enrico Zini \hfil {\tt enrico@enricozini.com}}

% Copyright: GNU GPL v2 or later.

\newcommand{\dballe}{{\tt DB-ALLe}}

\newcommand{\srcinsert}[1]{%
{\scriptsize
\verbatimtabinput[4]{#1}
}}

%\newlength{\savedmarginparwidth}
\newcommand{\figures}[2][!]{%
\smallskip
%\setlength{\savedmarginparwidth}{\marginparwidth}
%\setlength{\marginparwidth}{0pt}
\begin{minipage}[c]{\textwidth}
\ifthenelse{\equal{#1}{!}}{}{#1:\\}
#2
\end{minipage}
\medskip
%\setlength{\marginparwidth}{\savedmarginparwidth}
}

\makeindex

\begin{document}
\bibliographystyle{plain}

\maketitle

%\begin{abstract}
%This document is a quick introduction to the Fortran API of DB-ALLe.  It
%proceeds step by step to introducing different parts of the library using
%commented example code.
%\end{abstract}

\tableofcontents
%\listoffigures
%\listoftables

\chapter {Introduction}
\label{ch-intro}

\dballe{} is a fast, temporary, on-disk database where meteorological data can
be stored, searched, retrieved and updated.  To make computation easier, data
is stored as physical data, that is, as measures of a variable in a specific
point of space and time, rather than as a sequence of reports.

This is a quick introduction to \dballe{}.  It is intended as a quick read to
get up to speed, and as a quick reference for everyday use.

% Editorial notes

This guide contains examples for all routines, plus examples for all known
common usage patterns of dballe.  Also, commented Fortran code is preferred
to long explanations whenever it makes sense.

Other documentation can be found in \cite{UserGuide} and \cite{LibDoc}.

\chapter {Basic concepts}
\label{ch-concepts}

\section{Connections, sessions and handles}

\dballe{} stores meteorological values in a database.  This database can be
optionally shared or accessed using the network.

You contact \dballe{} by creating a \emph{connection}, and you work in
\dballe{} using one or more \emph{sessions}.  A connection is an established
link between your program and \dballe{}, and you usually open it at the
beginning of your work and close it at the end.

Within a connection you can create many working sessions.  This is very useful
because it allows to do different things at the same time, like reading a set
of values while writing computed results.

You can also set some safety features on sessions: for example, when you create
a session for reading values you can set it to disable all writes, which helps
you catch some programming mistakes.

You refer to the connection and the sessions using \emph{handles}.  A handle is
just an integer value that you use when you need to refer to something that is
not otherwise representable in any Fortran data type.

When you create a connection or a session, \dballe{} will give you the integer
handle that you will later use to refer to it.

\section {Pseudoana}

Values in \dballe{} are associated to informations about their source, such as
what kind of station generated the data and some characteristics of the station.

These informations are called \emph{pseudoana} informations, to suggest that
they are something similar to anagraphical informations about the sources of
the values.

A pseudoana entry is uniquely identified by:

\begin{itemize}
\item \emph{latitude}.
\item \emph{longitude}.
\item \emph{station identifier} (if the station is mobile).
\end{itemize}

\section {Values}

The main work of \dballe{} is to store and access \emph{measured values}.  A
measured value is described by:

\begin{enumerate}
\item The \emph{local B table descriptor}, which describes the nature of the
      variable, such as what it measures and the units used for measuring.
\item The measured \emph{value}.
\item \emph{pseudoana} information about what generated the value.
\item The vertical coordinates in space, as a description of the \emph{level layer}.
\item The \emph{time range} of the measurement.
\item The type of report (for example: measured value, forecast...).
\end{enumerate}

You can find a list of all current B table descriptors in \ref{btable}.

The type of report implicitly defines a priority of the measured value over
other equivalent values.  This can be used to select a ``best value'' from a
specific set of coordinates, where for example, values measured by synoptic
stations are preferred over values measured by a satellite, which in turn is
preferred over a value computed by a forecast model.

Priorities associated to the report type can be customized when creating the
database (see \ref{fun-idba_scopa}).

Every measured value in \dballe{} can be assigned any number of extra
attributes, which can be used, for example, for quality control.


\section {Attributes}

Values in \dballe{} are also associated to zero or more \emph{attributes}.  An
attribute is a value that describes an aspect of a variable, like for example a
confidence interval.

Attributes in \dballe{} consist on a value and a local B table descriptor,
which describes the nature of the variable, such as what it represents and the
units in which it is represented.

Attributes are uniquely identified by the variable they refer to and the B
table descriptor of the attribute.  As a consequence, you can only have one
attribute per value with a specific B table descriptor.  This means that, for
example, a variable can have at most one confidence interval attribute.


\section {Input, output, actions}
\label{basics-inout}

Work with \dballe{} happens using \emph{action routines}.  An action routine
typically reads some input, performs an action and produces some output.
Example of action routines are {\tt idba\_voglioquesto} to query data from a
database and {\tt idba\_prendilo} to write data into the database.

The input and the output of action routines are collections of parameters which
have a name and a value.  A list of parameters can be found in
\ref{parmtable}.

You can set the input parameters using the {\tt idba\_set*} functions:
\label{fun-idba_set}

\begin{description}
\item[{\tt idba\_seti(handle, "param", intvalue)}]
  Set the input parameter to the integer value {\tt intvalue}

\item[{\tt idba\_setc(handle, "param", charvalue)}]
  Set the input parameter to the character value {\tt charvalue}

\item[{\tt idba\_setr(handle, "param", realvalue)}]
  Set the input parameter to the real value {\tt realvalue}

\item[{\tt idba\_setd(handle, "param", doublevalue)}]
  Set the input parameter to the real*8 value {\tt doublevalue}
\end{description}

You can read the output parameters using the {\tt idba\_enq*} functions:
\label{fun-idba_enq}

\begin{description}
\item[{\tt idba\_enqi(handle, "param", intvalue)}]
  Read the output parameter into the integer value {\tt intvalue}

\item[{\tt idba\_enqc(handle, "param", charvalue)}]
  Read the output parameter into the character value {\tt charvalue}

\item[{\tt idba\_enqr(handle, "param", realvalue)}]
  Read the output parameter into the real value {\tt realvalue}

\item[{\tt idba\_enqd(handle, "param", doublevalue)}]
  Read the output parameter into the real*8 value {\tt doublevalue}
\end{description}

Note that all {\tt idba\_set} functions set input parameters, while all {\tt
idb\_enq*} functions read output parameters.  You cannot read input parameters
or set output parameters: that is the job of action routines.

In other words, input and output parameters are different things.  In this
code:

\begin{verbatim}
c     A possible misunderstanding
      call idba_seti(handle, "height", 1)
      call idba_enqi(handle, "height", val)
\end{verbatim}

the value of {\tt val} after the {\tt idba\_enqi} will not probably be 1, and
it could be either a value indicating ``missing value'' (in case no {\tt
height} parameter is set in the output parameters) or a {\tt height} value
previously retrieved by an action routine.

To reset one input parameter you can use {\tt idba\_unset}:
\label{fun-idba_unset}

\begin{verbatim}
c     ...
c     We don't want to limit results by latitude this time
      call idba_unset(handle, "latmin")
      call idba_unset(handle, "latmax")
      call idba_voglioquesto(handle, count)
\end{verbatim}

To reset all input parameters you can use {\tt idba\_unsetall}:
\label{fun-idba_unsetall}

\begin{verbatim}
c     Restart the query from scratch
      call idba_unsetall(handle)
      call idba_setd(handle, "latmin", 10.D0)
c     ...
\end{verbatim}

There is no way to reset output parameters: it is not needed since all action
routines will get rid of old output values before producing new ones.

In case one of the {\tt idba\_enq*} functions is called on a parameter which
does not exist, it will return a special value that indicates ``missing
value''.  This is a list of such special values:

\begin{tabular}{|l|l|}
\hline
{\em Data type} & {\em Missing value indicator}               \\
\hline
String  &  ""			\\
Int     & 0x7fffffff		\\
Real    & -1.1754944E-38	\\
Double  & -2.22507E-308		\\
\hline
\end{tabular}

\subsection {Input/output shortcuts}

There are a few functions that are shortcuts to other input and output
functions:

\begin{description}
\item[{\tt idba\_enqdate(handle, year, month, day, hour, minute, second)}]
  is a shortcut to:
  \begin{verbatim}
  idba_enqi(handle, "year", year)
  idba_enqi(handle, "month", month)
  idba_enqi(handle, "day", day)
  idba_enqi(handle, "hour", hour)
  idba_enqi(handle, "min", minute)
  idba_enqi(handle, "sec", second)
  \end{verbatim}
\item[{\tt idba\_setdate(handle, year, month, day, hour, minute, second)}]
  is a shortcut to:
  \begin{verbatim}
  idba_seti(handle, "year", year)
  idba_seti(handle, "month", month)
  idba_seti(handle, "day", day)
  idba_seti(handle, "hour", hour)
  idba_seti(handle, "min", minute)
  idba_seti(handle, "sec", second)
  \end{verbatim}
\item[{\tt idba\_enqlevel(handle, type, l1, l2)}]
  is a shortcut to:
  \begin{verbatim}
  idba_enqi(handle, "leveltype", type)
  idba_enqi(handle, "l1", l1)
  idba_enqi(handle, "l2", l2)
  \end{verbatim}
\item[{\tt idba\_setlevel(handle, type, l1, l2)}]
  is a shortcut to:
  \begin{verbatim}
  idba_seti(handle, "leveltype", type)
  idba_seti(handle, "l1", l1)
  idba_seti(handle, "l2", l2)
  \end{verbatim}
\item[{\tt idba\_enqtimerange(handle, type, p1, p2)}]
  is a shortcut to:
  \begin{verbatim}
  idba_enqi(handle, "pindicator", type)
  idba_enqi(handle, "p1", p1)
  idba_enqi(handle, "p2", p2)
  \end{verbatim}
\item[{\tt idba\_settimerange(handle, type, p1, p2)}]
  is a shortcut to:
  \begin{verbatim}
  idba_seti(handle, "pindicator", type)
  idba_seti(handle, "p1", p1)
  idba_seti(handle, "p2", p2)
  \end{verbatim}
\item[{\tt idba\_setcontextana(handle)}]
  is a shortcut to:
  \begin{verbatim}
  idba_seti(handle, "year", 1000)
  idba_seti(handle, "month", 1)
  idba_seti(handle, "day", 1)
  idba_seti(handle, "hour", 0)
  idba_seti(handle, "min", 0)
  idba_seti(handle, "sec", 0)
  idba_seti(handle, "leveltype", 257)
  idba_seti(handle, "l1", 0)
  idba_seti(handle, "l2", 0)
  idba_seti(handle, "pindicator", 0)
  idba_seti(handle, "p1", 0)
  idba_seti(handle, "p2", 0)
  idba_seti(handle, "rep_cod", 254)
  \end{verbatim}
\end{description}

\section {API invariants}

\begin{itemize}
\item All errors are reported as function return values
\item All results are reported as output parameters
\item All functions {\tt idba\_set*} set the input of action routines
\item All functions {\tt idba\_enq*} get the output of action routines
\end{itemize}

\section {Error management}

Errors can be handled by checking the return value of every function:

\begin{verbatim}
c     Example error handling
      ierr = idba_presentati(dbhandle, "dsn", "user", "password")
      if (ierr.ne.0) then
c       handle the error...
      end if
\end{verbatim}

Or they can be handled by installing a callback function that is automatically
called in case of error:

\begin{verbatim}
c     How to set a callback
c       * the first parameter is the error code that triggers the callback (0
c         means 'trigger on all errors')
c       * the second parameter is the routine to call when the error happens        
c         (remember to declare the function as 'external'
c       * the third parameter is a convenience arbitrary integer that will be
c         passed to the function
c       * the fourth parameter is used to return a handle that can be used to
c         remove the callback
      call idba_error_set_callback(0, error_handler, 42, cb_handle)
\end{verbatim}

The previous code will setup DB-ALLe to call {\tt error\_handler} after any error,
passing it the integer value 42.  The callback can be removed at any time by
calling {\tt idba\_error\_remove\_callback}:
 
\begin{verbatim}
c     How to remove a callback
      call idba_error_remove_callback(cb_handle)
\end{verbatim}

This is a useful error handling function:

\begin{verbatim}
c     The error handler needs to be declared 'external'
      external error_handler

c     Compute the length of a string
c     (this is an utility function that is used by the error handler
c      to print nicer error messages)
      integer function istrlen(string)
      character string*(*)
        istrlen = len(string)
        do while ((string(istrlen:istrlen).eq." " .or.
     $       string(istrlen:istrlen).eq."").and.
     $       istrlen.gt.0)
           istrlen = istrlen - 1
        enddo
        return
      end

c     Generic error handler: print all available information
c     about the error, then exit
      subroutine error_handler(val)
      integer val
      character buf*1000
        print *,ier," testcb in ",val
        call idba_error_message(buf)
        print *,buf(:istrlen(buf))
        call idba_error_context(buf)
        print *,buf(:istrlen(buf))
        call idba_error_details(buf)
        print *,buf(:istrlen(buf))
        call exit (1)
        return
      end
\end{verbatim}

This code introduces three new functions:

\begin{description}
\item[{\tt idba\_error\_message}]
  returns a string describing what type of error has happened.
\item[{\tt idba\_error\_context}]
  returns a string describing what \dballe{} was trying to do when the error
  happened.
\item[{\tt idba\_error\_details}]
  returns a detailed description of the error, when available.  If no detailed
  description is available, it returns an empty string.
\end{description}



\chapter {A working session}
\label{ch-work}

\section{Starting the work}

Before any action routine, you need to connect to the database.  Connecting to
the database will give you a \emph{handle} that you can use to open sessions.

This code will open a connection with \dballe{}, then it will start a session:

\label{fun-idba_presentati}
\label{fun-idba_preparati}

\begin{verbatim}
c     Connect to the database and get a handle to work with it
      call idba_presentati(dbhandle, "dsn", "username", "password")
      call idba_preparati(dbhandle, handle, "read", "read", "read")
\end{verbatim}

You can call {\tt idba\_preparati} many times and get more handles.  This allows
to perform many operations on the database at the same time.

{\tt idba\_preparati} has three extra parameters that can be used to limit
write operations on the database, as a limited protection against programming
errors.

The first extra parameter controls access to pseudoana records and can have
these values:

\begin{description}
\item[{\tt "read"}] pseudoana records cannot be modified.
\item[{\tt "write"}] pseudoana records can be added and removed.
\end{description}

The second extra parameter control access to observed data and can have
these values:

\begin{description}
\item[{\tt "read"}] data cannot be modified in any way.
\item[{\tt "add"}] data can be added to the database, but existing data cannot
		   be modified.  Deletions are disabled.  This is used to
		   insert new data in the database while preserving the data
		   that was already present in it.
\item[{\tt "write"}] data can freely be added, overwritten and deleted.
\end{description}

The third extra parameter controls access to data attributes and can have
these values:

\begin{description}
\item[{\tt "read"}] attributes cannot be modified in any way.
\item[{\tt "add"}] attributes can can be added to the database, but existing
		   attributes cannot be modified.  Deletion of attributes is
		   disabled.  This is used to insert new attribute in the
		   database while preserving the attributes that were already
		   present in it.
\item[{\tt "write"}] attributes can freely be added, overwritten and deleted.
\end{description}

Note that some combinations are illegal, such as pseudoana=read and data=add
(when adding a new data, it's sometimes necessary to insert new pseudoana
records), or data=rewrite and attr=read (when deleting data, their attributes
are deleted as well).


\section{Setting input and reading output}

Input to action routines is done using the functions {\tt idba\_set*}, and output
is read with the functions {\tt idba\_enq*} (see \ref{basics-inout}):

\begin{verbatim}
c     Set the extremes of an area and retrieve all the stations in it
      call idba_setd(handle, "latmin", 30.D0)
      call idba_setd(handle, "latmax", 50.D0)
      call idba_setd(handle, "lonmin", 10.D0)
      call idba_setd(handle, "lonmax", 20.D0)
      call idba_quantesono(handle, count)
   
c     Get the informations about a station
      do while (count.gt.0)
        call idba_elencamele(handle)
        call idba_enqc(handle, "name", cname)
        call idba_enqi(handle, "ana_id", id)
        call idba_enqd(handle, "lat", lat)
        call idba_enqd(handle, "lon", lon)
c       ....
        count = count - 1
      enddo
\end{verbatim}

Note that, when one uses {\tt idba\_setc}, {\tt idba\_seti}, {\tt idba\_enqc},
{\tt idba\_enqi} with parameters that have some decimal digits, \dballe{} will
work with values as if they did not have a decimal point.  That is, if latitude
$10.124323$ is read with {\tt idba\_enqi}, then the result will be $10124323$.

The following example shows exactly what happens:

\begin{verbatim}
c     Set the latitude to 30.0 degrees
      call idba_setr(handle, "lat", 30.0)
c     Set the latitude to 30.0 degrees
      call idba_setd(handle, "lat", 30.0D0)

c     Set the latitude to 0.00030 degrees
      call idba_seti(handle, "lat", 30)
c     Set the latitude to 30.0 degrees
      call idba_seti(handle, "lat", 3000000)

c     Set the latitude to 0.00030 degrees
      call idba_setc(handle, "lat", "30")
c     Set the latitude to 30.0 degrees
      call idba_setc(handle, "lat", "3000000")
\end{verbatim}

\subsection {Parameter names}

There are three different kinds of parameter names one can use: the \dballe{}
parameters, the WMO B variables and the variable aliases.

The \dballe{} parameters, fully described in \ref{parmtable}, are parameters
that have a special meaning to \dballe{}: for example they can be part of the
coordinate system, or space/time extremes to use to query the database.  They
are indicated simply with their name (for example, {\tt "lat"} or {\tt
"yearmin"}).

The WMO B variables, fully described in the WMO B table, represent all possible
sorts of observed data, and are indicated in the form ``B{\itshape xxyyy}'',
where {\itshape xxyyy} are the X and Y values from the WMO table B.  For more
details on working with observed data, see \ref{obsdata}.  For a list of all
the currently available B table descriptors, see \ref{btable}.

The variable aliases are short, easy to remember names which can be used
instead of frequently used WMO B variables.  For a list of all the currently
available aliases, see \ref{aliastable}.

\subsection{Queries and observed data}
\label{obsdata}

The {\tt idba\_set*} and {\tt idba\_enq*} functions can also be used to set and get
observation data.  To do so, use as parameter the string {\tt "Bxxyyy"}, where
{\tt xx} and {\tt yyy} are the X and Y values of the BUFR/CREX table B describing
the observed data.

For example:

\begin{verbatim}
c     Set the speed of the wind (very useful in summer)
      call idba_setr(handle, "B11002", 1.8)
c     Also set the temperature
      call idba_setr(handle, "B12001", 21.8)
      call idba_prendilo(handle)
\end{verbatim}


\subsection{Attributes}

The {\tt idba\_set} and {\tt idba\_enq} groups of functions can also be used to
set and get attributes on data.  To do so, use as parameter the string {\tt
"*Bxxyyy"}, where {\tt xx} and {\tt yyy} are the X and Y values of the
BUFR/CREX table B describing the attribute.

For example:

\begin{verbatim}
c     Set the confidence of the wind speed value we inserted
c     in the last 'idba_prendilo'
      call idba_setr(handle, "*B33007", 75.0)
      call idba_setc(handle, "*var_related", "B11002")
      call idba_critica(handle)
\end{verbatim}

\section {Querying the database}

Queries are made by giving one or more extremes of space, time, level or time
range.  You can refer to section \ref{parmtable} for a list of all available
query parameters (see the column ``On query input'').

\subsection {Querying the anagraphical data}

Example code to query all the stations in a given area:
\label{fun-idba_quantesono}
\label{fun-idba_elencamele}

\begin{verbatim}
      call idba_setd(handle, "latmin", 30.D0)
      call idba_setd(handle, "latmax", 50.D0)
      call idba_setd(handle, "lonmin", 10.D0)
      call idba_setd(handle, "lonmax", 20.D0)
      call idba_quantesono(handle, count)
      do while (count.gt.0)
        call idba_elencamele(handle)
        call idba_enqc(handle, "name", cname)
        call idba_enqi(handle, "ana_id", id)
c       query more data and work with it
        count = count - 1
      enddo
\end{verbatim}

This code introduces two new functions:

\begin{description}
\item[{\tt idba\_quantesono}]
  performs the query and returns the number of stations it finds.

\item[{\tt idba\_elencamele}]
  gets a station out of the results of {\tt idba\_quantesono}.  If there are no
  more stations, the function fails.
\end{description}

\subsection {Querying the values}

Example code to query all the values in a given area and time:
\label{fun-idba_voglioquesto}
\label{fun-idba_dammelo}

\begin{verbatim}
      call idba_seti(handle, "latmin", 30)
      call idba_seti(handle, "latmax", 50)
      call idba_seti(handle, "lonmin", 10)
      call idba_seti(handle, "lonmax", 20)
      call idba_seti(handle, "yearmin", 2004)
      call idba_seti(handle, "yearmax", 2004)
      call idba_voglioquesto(handle, count)
      do while (count.gt.0)
        call idba_dammelo(handle, param)
c       get the value of this variable
        call idba_enqc(handle, param, cvalue)
        call idba_enqd(handle, "lat", dlat)
        call idba_enqd(handle, "lon", dlon)
c       query more data and work with it
        count = count - 1
      enddo
\end{verbatim}
   
This code introduces two new functions:

\begin{description}
\item[{\tt idba\_voglioquesto}]
  performs the query and returns the number of values it finds.
\item[{\tt idba\_dammelo}]
  gets a value out of the result of {\tt idba\_voglioquesto}.  If there are no
  more stations, the function fails.
\end{description}

\subsection {Clearing the database}

You can initialise or reinitialise the database using {\tt idba\_scopa}:
\label{fun-idba_scopa}

\begin{verbatim}
c     Start the work with a clean database
      call idba_scopa(handle, "repinfo.csv")
\end{verbatim}

{\tt idba\_scopa} clears the database if it exists, then recreates all the
needed tables.  Finally, it populates the informations about the reports (such
as the available report types, their mnemonics and their priority) using the
data in the file given as argument.

The file is in CSV format, with 6 columns:
\begin{enumerate}
\item Report code (corresponding to parameter {\tt rep\_cod})
\item Mnemonic name (corresponding to parameter {\tt rep\_memo})
\item Report description
\item Report priority (corresponding to parameter {\tt priority})
\item Ignored
\item Ignored
\end{enumerate}

If {\tt ""} is given instead of the file name, {\tt idba\_scopa} will read the
data from {\tt /etc/repinfo.csv}.

This is an example of the contents of the file:

\begin{verbatim}
01,synop,report synottico,100,oss,0
02,metar,metar,80,oss,0
03,temp,radiosondaggio,100,oss,2
04,ana_lm,valori analizzati LM,-1,ana,255
05,ana,analisi,-10,pre,255
06,pre_cleps_box1.5maxel001,previsti cosmo leps box 1.5 gradi valore max elemento 1,-1,pre,255
07,pre_lmn_box1.5med,previzione Lokal Model nudging box 1.5 gradi valore medio,-1,pre,255
08,pre_lmp_spnp0,previsione Lkal Model prognostica interpolato punto piu' vicino,-1,pre,255
09,boe,dati omdametrici,100,oss,31
\end{verbatim}

{\tt idba\_scopa} will not work unless {\tt rewrite} has been enabled for the
data when opening the database.


\subsection {Inserting data}

Data is inserted using {\tt idba\_prendilo}:
\label{fun-idba_prendilo}
         
\begin{verbatim}
c     Insert a new data in the database
      call idba_setr(handle, "ana_id", 4)
      call idba_setr(handle, "rep_memo", "synop")
      call idba_setd(handle, "lat", 44.500D0)
      call idba_setd(handle, "lon", 11.328D0)
      call idba_setr(handle, "year", 2005)
      call idba_setr(handle, "month", 7)
      call idba_setr(handle, "day", 26)
      ...
      call idba_setr(handle, "B11002", 1.8)
      call idba_prendilo(handle)
\end{verbatim}

This code introduces a new function:

\begin{description}
\item[{\tt idba\_prendilo}]
  inserts a new value in the database.  All the information about the parameter
  to insert is taken from the input previously set by {\tt idba\_set*} functions.

  When data of the same kind and with the same characteristics already exists,
  the behaviour of {\tt idba\_prendilo} is defined by the parameter passed to
  {\tt idba\_preparati} when creating the handle.  See `Starting the work`\_ for
  more informations.
\end{description}

{\tt idba\_prendilo} will work in different ways according to the data opening
mode of the database:

\begin{description}
\item[{\tt read}]
  causes an error, because the data cannot be read.
\item[{\tt add}]
  new data can be inserted, but causes an error when trying to insert a value
  that already exists.
\item[{\tt rewrite}]
  new data can be inserted, and existing data is overwritten.
\end{description}

Also, behaviour changes according to the pseudoana opening mode:

\begin{description}
\item[{\tt "reuse"}] when inserting data, if an existing pseudoana record for
	       the data is found, it will be reused.
\item[{\tt "rewrite"}] when inserting data, if an existing pseudoana record for
	     the data is found, it will be completely overwritten with
	     the parameters in input.
\end{description}

Note that the database cannot be opened in pseudoana {\tt read} mode when data
is {\tt add} or {\tt rewrite}.
  
\section{Deleting data}

Data is deleted using {\tt idba\_dimenticami}:
\label{fun-idba_dimenticami}

\begin{verbatim}
c     Delete all data from the station with id 4 in year 2002
      call idba_seti(handle, "ana_id", 4)
      call idba_seti(handle, "year", 2002)
      call idba_dimenticami(handle)
\end{verbatim}

This code introduces a new function:

\begin{description}
\item[{\tt idba\_dimenticami}]
  deletes all the data found in the extremes specified in input.
\end{description}

{\tt idba\_dimenticami} will not work unless {\tt rewrite} has been enabled for
the data when opening the database.

\section {Attributes}

\subsection{Reading attributes}

Attributes are read using {\tt idba\_ancora}:
\label{fun-idba_voglioancora}
\label{fun-idba_ancora}

\begin{verbatim}
c     ...setup a query...
      idba_voglioquesto(handle, count)
      do while (count.gt.0)
        call idba_dammelo(handle, param)

c       Read QC informations about the last value read
        call idba_voglioancora(handle, qc_count)
        do while (qc_count.gt.0)
            call idba_ancora(handle, param) 
            call idba_enqc(handle, param, value)
c           ...process the value...
            qc_count = qc_count - 1
        enddo

        count = count - 1
      enddo
\end{verbatim}

This code introduces two new functions:

\begin{description}
\item[{\tt idba\_voglioancora}]
  Performs a query to retrieve attributes for the last variable read by
  {\tt idba\_dammelo}.  It returns the number of attributes available.
\item[{\tt idba\_ancora}]
  Retrieves one by one the values queried by {\tt idba\_voglioancora} if
  there are no more items available, the function will fail.

  The parameter {\tt param} will be set to the name (in the form {\tt *Bxxyyy}) of
  the attribute just read.
\end{description}

\subsection{Writing attributes}

\label{fun-idba_critica}

Attributes are written using {\tt idba\_critica}, which can be used after an
{\tt idba\_dammelo}, after an {\tt idba\_prendilo} or at any time using a stored data
id.  These three case differ on how to communicate to {\tt idba\_critica} what is
the data about which to write attributes.

When used after {\tt idba\_dammelo}, {\tt idba\_critica} can refer directly to the
last data retrieved:

\begin{verbatim}
c     ...setup a query...
      call idba_voglioquesto(handle, count)
      do while (count.gt.0)
        call idba_dammelo(handle, param)
c       ...process data...

c       Set the attributes
        call idba_seti(handle, "*B33007", 75)
        call idba_seti(handle, "*B33006", 42)
        call idba_critica(handle)

        count = count - 1
      enddo
\end{verbatim}

After an {\tt idba\_prendilo} instead, since {\tt idba\_prendilo} can write more than
one data at a time, we need to tell {\tt idba\_critica} which of them we are
referring to:

\begin{verbatim}
c     Insert wind speed and temperature
      call idba_setr(handle, "B11002", 1.8)
      call idba_setr(handle, "B12001", 22)
      call idba_prendilo(handle)

c     Set the attributes
      call idba_seti(handle, "*B33007", 75)

c     Use "*var_related" to indicate which of the two variables we are annotating
      call idba_setc(handle, "*var_related", "B11002")

      call idba_critica(handle)
\end{verbatim}


{\tt idba\_critica} can also be called at any time using a previously stored data it:

\begin{verbatim}
c     ...perform a query with idba\_voglioquesto...
      do while (count.gt.0)
        call idba_dammelo(handle, param)
c       ...process data...

c       This variable is interesting: save the context ID
c       to refer to it later
        call idba_enqi(handle, "context_id", saved_id)
        
        count = count - 1
      enddo

c     ...some time later...

c     Insert attributes about that interesting variable
      call idba_seti(handle, "*B33007", 75)
      call idba_seti(handle, "*B33006", 42)
      
c     Select the variable using its context id
c     and variable code
      call idba_seti(handle, "*context_id", saved_id)
      call idba_seti(handle, "*var_related", "B11001")
      call idba_critica(handle)
\end{verbatim}

This code introduces a new function:

\begin{description}
\item[{\tt idba\_critica}]
  Set one or more attributes about a variable.
  
  The variable can be identified directly by using {\tt idba\_seti(handle,
  "*context\_id", id)} and {\tt idba\_seti(handle, "*var\_related", name)}.
  These parameters are automatically set by the {\tt idba\_dammelo} and {\tt
  idba\_prendilo} action routines.

  The attributes and values are set as input to {\tt idba\_critica} using the
  {\tt idba\_set*} functions with an asterisk in front of the variable name.
\end{description}


{\tt idba\_critica} will work in different ways according to the attributes
opening mode of the database:

\begin{description}
\item[{\tt "read"}] attributes cannot be modified in any way.
\item[{\tt "add"}] attributes can can be added to the database, but existing
		   attributes cannot be modified.
\item[{\tt "rewrite"}] attributes can be added, and existing attributes can be
                   overwritten.
\end{description}


\subsection{Deleting attributes}

Attributes are deleted using {\tt idba\_scusa}:

\label{fun-idba_scusa}

\begin{verbatim}
c     Delete the confidence interval from the wind speed

c     The referring variable is identified in the same way as with
c     idba_critica:
      call idba_seti(handle, "*context_id", saved_id)
      call idba_seti(handle, "*var_related", "B11002")

c     The attributes to delete are selected by setting "*varlist":
      call idba_setc(handle, "*varlist", "*B33007")
      call idba_scusa(handle)
\end{verbatim}

This code introduces a new function:

\begin{description}
\item[{\tt idba\_scusa}]
  Delete attributes from a variable identified in the same way as with {\tt
  idba\_critica}.  By default it deletes all attributes; specific attributes to
  delete can be selected by setting their names, separated by commas, in the
  "*varlist" variable.
\end{description}

{\tt idba\_scusa} will not work unless the database has been opened in
attribute {\tt rewrite} mode.


\section{Ending the work}

When you are finished working with a handle, you can release it with {\tt idba\_fatto}

\label{fun-idba_fatto}

\begin{verbatim}
c     We are finished with this handle
      call idba_fatto(handle)
\end{verbatim}

When you are finished working with DB-ALLe, you can ask it to close all
connections and release all resources:

\label{fun-idba_arrivederci}

\begin{verbatim}
c     We do not need to work with dballe anymore
      call idba_arrivederci(dbh)
\end{verbatim}

\chapter{Advanced topics}

\section{Shortcuts to stations and data}

\dballe{} offers two shortcuts to represent pseudoana entries and data in the
database: the {\tt ana\_id} and the {\tt data\_id} keys, that are set in the
output of every {\tt idba\_dammelo}.

{\tt ana\_id} represents a pseudoana entry.  Every time one needs to specify a
set of latitude, longitude, fixed/mobile, one could use the corresponding {\tt
ana\_id} value, if known, and get a faster search.

{\tt data\_id} represents a data entry.  Every time one needs to identify some
data setting latitude, longitude, level layer, time range and so on, one can
just provide the value of {\tt data\_id}, and also get a faster search.

\section{Query modifiers}

\dballe{} allows to set special query behaviours using the {\tt "query"}
parameter.  The available options are:

\begin{description}
\item [best]
  When measures from different kinds of reports exist in the same physical
  space, do not return them all, but only return the one of the record type
  with the highest priority.
\item [bigana]
  Use special query optimizations that make things faster when there are a lot
  of stations in the database, and slower otherwise.  If you are having slow
  queries, try setting this option and see if it helps.
\item [nosort]
  Return the results in a nondefined order.  It can make things faster when
  writing algorithms that do not depend on the ordering of data from \dballe{}.
\item [stream]
  Normally the functions {\tt idba\_quantesono} or {\tt idba\_voglioquesto}
  will read all the results from the server, count them and save them locally.

  When using this option, those functions will return right after performing
  the query, and the results will be read later using {\tt idba\_elencamele} or
  {\tt idba\_dammelo}.  As a side effect, {\tt idba\_quantesono} and
  {\tt idba\_voglioquesto} cannot know in advance the result count, and will
  return some undefined value in it.

  When using this option, the only way to find out when to stop reading is to
  know that {\tt idba\_elencamele} and {\tt idba\_dammelo} will set all the
  output parameters to undefined when there is no more data to read.  One could
  thus use a loop like this:
  \begin{verbatim}
      call idba_setc (handle, "query", "stream")
c     Make the query
      call idba_voglioquesto (handle, N)
c     Iterate the results.  Ignore N because it is
c     meaningless when using query=stream
      do
        call idba_dammelo (handle, varname)
c       If ana_id is missing value, then we are done
        idba_enqi(handle, "ana_id", value)
        if (value .eq. 2147483647) exit

c       ...process the data...
      enddo
  \end{verbatim}

  This option can be a significant improvement in speed and memory when the
  result count is not needed in advance.  The speed improvement comes because
  the client can start doing computations on the results while the server is
  still generating the results, taking advantage of parallelism, especially if
  the server and the application are on two different computers.

  Without using this option, the client cannot start to use the results until
  the server has finished computing and trasmitting them.
\end{description}

\chapter{Code examples}
\label{ch-examples}

% TODO: put some more example code snippets from real-world

\section{Insert station data, then insert data}

\begin{verbatim}
      call idba_preparati(dbhandle, handle, "write", "add", "add")
      
c     Insert data about a station
      call idba_seti (handle, "!ana", "")
      call idba_setr (handle, "lat", 11.345)
      call idba_setr (handle, "lon", 44.678)
      call idba_setr (handle, "height", 23)
      call idba_prendilo (handle)

c     Read the station ID for the station we just inserted
      call idba_enqi (handle, "!ana_id", anaid)

c     Reset the input data
      call idba_unsetall (handle)

c     Add data to the station we just inserted
      call idba_seti (handle, "ana_id", anaid)
      call idba_setlevel (handle, 100, 1, 0)
      call idba_settimerange (handle, 0, 0, 0)
      call idba_setdate (handle, 2006, 06, 20, 19, 30, 0)
      call idba_seti (handle, "t", 21)
      call idba_setc (handle, "B12345", "ciao")
      call idba_prendilo (handle)
\end{verbatim}


\section{Query data, then query station data}

\begin{verbatim}
      call idba_preparati(dbhandle, handle, "read", "read", "read")
      call idba_preparati(dbhandle, handleana, "read", "read", "read")
      
c     Prepare a query
      call idba_setd (handle, "latmin", 10)
      ...
      call idba_setd (handle, "lonmax", 60)

c     Make the query
      call idba_voglioquesto (handle, N)

c     Iterate the results
      do i=1,N
        call idba_dammelo (handle, varname)

c       Read data about the variable we just had
        call idba_enqlevel (handle, ltype, l1, l2)

c       Read pseudoana data about the variable we just had
c       Setup a query for the station with 'quantesono'
        call idba_enqi (handle, "ana_id", anaid)
        call idba_seti (handleana, "ana_id", anaid)

c       Query.  Nstaz should always be 1 because we query a specific station
        call idba_quantesono (handleana, Nstaz)

c       Fetch the data
        call idba_elencamele (handleana)

c       Read the data about the station
        call idba_enqi (handleana, "height", height)
      enddo
\end{verbatim}


\chapter{FAQ and Troubleshooting}
\label{ch-trouble}

\begin{description}
\item [ {\tt call idba\_setc (handle, "query", "stream")} does not make any difference ]
  Because of a quirk of the MySQL ODBC driver, you may need to add a special
  configuration option to the MySQL driver.
  
  These are the instructions that you, or the person who set up ODBC in your
  computer, can follow:
  \begin{quote}
  You need to add 1048576 to the value of the Options
  field of the configuration of the ODBC DSN.

  This can be done by adding the line {\tt Option = 1048576} to the file {\tt
  odbc.ini}, or editing the Options field using the graphical ODBC
  configuration tool.
  \end{quote}
\end{description}

\appendix

\chapter{Input and output parameters}

\section{For data-related action routines}

\label{parmtable}
\input{../dballe/core/record_keyword.tex}

\section{For attribute-related action routines}

{\begin{scriptsize}
\begin{longtable}{|@{\hspace{0.5mm}}l@{\hspace{0.5mm}}|@{\hspace{0.5mm}}l@{\hspace{0.5mm}}|@{\hspace{0.5mm}}l@{\hspace{0.5mm}}|@{\hspace{0.5mm}}p{2.0cm}@{\hspace{0.5mm}}|@{\hspace{0.5mm}}p{1.4cm}@{\hspace{0.5mm}}|@{\hspace{0.5mm}}p{1.4cm}@{\hspace{0.5mm}}|@{\hspace{0.5mm}}p{1.4cm}@{\hspace{0.5mm}}|@{\hspace{0.5mm}}p{3.5cm}@{\hspace{0.5mm}}|}
\hline
{\em Name} & {\em Unit} & {\em Format} & {\em Description} & {\em On insert input} & {\em On query input} & {\em On output} & {\em Comment} \\
\hline
\endhead
\hline
\endfoot
{\tt *Bxxyyy} & depends & depends & Value of the attribute & required & ignored & present & \\
{\tt *var} & Character & 7 chars & Attribute queried & ignored & optional & present, indicates the name of the last attribute returned &  \\
{\tt *varlist} & Character & 255 chars & List of attributes to query & ignored & optional & absent & Comma-separated list of attribute B codes wanted on output \\
{\tt *var\_related} & Character & 6 chars & Variable related to the attribute to query & required & required & absent & It is automatically set by {\tt idba\_dammelo} and {\tt idba\_prendilo} (when {\tt idba\_prendilo} inserts a single variable) \\
{\tt *context\_id} & Numeric & 10 digits & Context ID of the variable related to the attribute to query & required & required & absent & It is automatically set by {\tt idba\_dammelo} and {\tt idba\_prendilo} \\
\hline
\end{longtable}
\end{scriptsize}
}


\chapter{Values for level layer and time range}

\section{Level layer}

This table lists all the various value combinations to use as a level layer:

\begin{scriptsize}
\begin{longtable}{|@{\hspace{0.5mm}}l@{\hspace{0.5mm}}|@{\hspace{0.5mm}}p{4.0cm}@{\hspace{0.5mm}}|@{\hspace{0.5mm}}p{4.0cm}@{\hspace{0.5mm}}|@{\hspace{0.5mm}}p{4.0cm}@{\hspace{0.5mm}}|}
\hline
{\em Level type} & {\em Meaning} & {\em Contents ({\tt l1})} & {\em Contents ({\tt l2})} \\
\hline
\endhead
\hline
\endfoot
  0     & Reserved & & \\
  1     & Ground or water surface & 0 & \\
  2     & Cloud base level & 0 & \\
  3     & Level of cloud tops & 0 & \\
  4     & Level of 0$^o$C isotherm & 0 & \\
  5     & Level of adiabatic condensation lifted from the surface & 0 & \\
  6     & Maximum wind level & 0 & \\
  7     & Tropopause & 0 & \\
  8     & Nominal top of atmosphere & 0 & \\
  9     & Sea bottom & 0 & \\
10-19   & Reserved & & \\
20      & Isothermal level & Temperature in 1/100 K & \\
21-99   & Reserved & & \\
100     & Isobaric surface & Pressure in hPa (2 octets) & \\
101     & Layer between two isobaric surfaces & Pressure of top in kPa & Pressure of bottom in kPa \\
102     & Mean sea level & 0 & \\
103     & Specified altitude above mean sea level  & Altitude in meters (2 octets) & \\
104     & Layer between two specified altitudes above mean sea level & Altitude of top in hm & Altitude of bottom in hm \\
105     & Specified height above ground & Height in meters (2 octets) & \\
106     & Layer between two specified height levels above ground & Height of top in hm & Height of bottom in hm \\
107     & Sigma level & Sigma value in 1/10000 (2 octets) & \\
108     & Layer between two sigma levels & Sigma value of top in 1/100 & Sigma value of bottom in 1/100 \\
109     & Hybrid level & Level number (2 octets) & \\
110     & Layer between two hybrid levels & Level number of top & Level number of bottom \\
111     & Depth below land surface & Depth in centimeters (2 octets) & \\
112     & Layer between two depths below land surface & Depth of upper surface in cm & Depth of lower surface in cm \\
113     & Isentropic (theta) level & Potential temperature in K (2 octets) & \\
114     & Layer between two isentropic levels & 475 K minus theta of top in K & 475 K minus theta of bottom in K \\
115     & Level at specified pressure difference from ground to level & Pressure difference in hPa (2 octets) & \\
116     & Layer between two levels at specified pressure differences from ground to level & Pressure difference from ground to top level in hPa & Pressure difference from ground to bottom level in hPa \\
117     & Potential vorticity surface & 10-9 K m2 kg-1 s-1 & \\
118     & Reserved & & \\
119     & ETA* level & ETA value in 1/10000 (2 octets) & \\
120     & Layer between two ETA* levels & ETA value at top of layer in 1/100 & ETA value at bottom of layer in 1/100 \\
121     & Layer between two isobaric surfaces (high precision) & 1100 hPa minus pressure of top in hPa & 1100 hPa minus pressure of bottom in hPa \\
122-124 & & & \\
125     & Specified height level above ground (high precision) & Height in centimeters (2 octets)  & \\
126-127 & Reserved & \\
128     & Layer between two sigma levels (high precision) & 1.1 minus sigma of top in 1/1000 of sigma & 1.1 minus sigma of bottom in 1/1000 of sigma \\
129-140 & & & \\
141     & Layer between two isobaric surfaces (mixed precision) & Pressure of top in kPa & 1100 hPa minus pressure of bottom in hPa \\
142-159 & Reserved & & \\
160     & Depth below sea level & Depth in meters (2 octets) & \\
161-199 & Reserved & & \\
200     & Entire atmosphere (considered as a single layer) & 0 & \\
201     & Entire ocean (considered as a single layer) & 0 & \\
202-254 & Reserved & & \\
256	& Cloud levels       & 0 & 0=General Cloud Group, 1=CL, 2=CM, 3=CH \\
256	& Synop cloud group 1 & 1 & 0 \\
256	& Synop cloud group 2 & 2 & 0 \\
256	& Synop cloud group 3 & 3 & 0 \\
256	& Synop cloud group 4 & 4 & 0 \\
257	& Extra anagraphical informations & & \\
\hline
\end{longtable}
\end{scriptsize}



\section{Time range}

This table lists all the various value combinations to use as a time range:

\begin{description}
\item [0]
  Forecast product valid for reference time + P1 (P1 > 0), or Uninitialized
  analysis product for reference time (P1 = 0), or Image product for reference
  time (P1 = 0).
\item [1]
  Initialized analysis product for reference time (P1 = 0).
\item [2]
  Product with a valid time ranging between reference time + P1 and reference time + P2.
\item [3]
  Average (reference time + P1  to reference time + P2).
\item [4]
  Accumulation (reference time + P1 to reference time + P2) product
  considered valid at reference time + P2.
\item [5]
  Difference (reference time + P2 minus reference time + P1) product
  considered valid at reference time + P2.
\item [6-9] Reserved.
\item [10]
  P2 = 0; product valid at reference time + P1.
\item [11-50] Reserved.
\item [51]
  Climatological Mean Value: multiple year averages of quantities which are
  themselves means over some period of time (P2) less than a year. The
  reference time (R) indicates the date and time of the start of a period of
  time, given by R to R + P2 , over which a mean is formed; N indicates the
  number of such period-means that are averaged together to form the
  climatological value, assuming that the N period-mean fields are separated by
  one year. The reference time indicates the start of the N-year climatology. N
  is given in octets 22 and 23 of the PDS.

  If P1 = 0, then the data averaged in the basic interval P2 are assumed to be
  continuous (i.e., all available data are simply averaged together).

  If P1 = 1 (the units of time - octet 18,Code table 4 - are not relevant
  here), then the data averaged together in the basic interval P2 are valid
  only at the time (hour, minute) given in the reference time, for all the days
  included in the P2 period. The units of P2  are given by the contents of
  octet 18 and Code table 4.
\item [52-112] Reserved.
\item [113]
  Average of N forecasts (or initialized analyses); each product has forecast
  period of P1 (P1 = 0 for initialized analyses); products have reference
  times at intervals of P2, beginning at the given reference time.
\item [114]
  Accumulation of N forecasts (or initialized analyses); each product has
  forecast period of P1 (P1 = 0 for initialized analyses); products have
  reference times at intervals of P2, beginning at the given reference time.
\item [115]
  Average of N forecasts, all with the same reference time; the first has a
  forecast period of P1, the remaining forecasts follow at intervals of P2.
\item [116]
  Accumulation of N forecasts, all with the same reference time; the first has
  a forecast period of P1, the remaining forecasts follow at intervals of P2.
\item [117]
  Average of N forecasts; the first has a forecast period of P1, the
  subsequent ones have forecast periods reduced from the previous one by an
  interval of P2; the reference time for the first is given in octets 13 to 17,
  the subsequent ones have reference times increased from the previous one by
  an interval of P2. Thus all the forecasts have the same valid time, given by
  the reference time + P1.
\item [118]
  Temporal variance, or covariance, of N initialized analyses; each product has
  forecast period of P1  = 0; products have reference times at intervals of P2,
  beginning at the given reference time.
\item [119]
  Standard deviation of N forecasts, all with the same reference time with
  respect to the time average of forecasts; the first forecast has a forecast
  period of P1, the remaining forecasts follow at intervals of P2.
\item [120-122] Reserved.
\item [123]
  Average of N uninitialized analyses, starting at the reference time, at
  intervals of P2.
\item [124]
  Accumulation of N uninitialized analyses, starting at the reference time, at
  intervals of P2.
\item [125-254] Reserved.
\item [256]
  When P1=0 and P2=0: General cloud group \\
  When P1=0 and P2=1: CL \\
  When P1=0 and P2=2: CM \\
  When P1=0 and P2=3: CH \\
  When P1=1 and P2=0: Vertical significance 1 \\
  When P1=2 and P2=0: Vertical significance 2 \\
  When P1=3 and P2=0: Vertical significance 3 \\
  When P1=4 and P2=0: Vertical significance 4 \\
  When P1=0 and P2=0: Extra station information
\end{description}

Notes about the time range values:

\begin{itemize}
\item For analysis products, or the first of a series of analysis products, the reference time (octets 13 to 17) indicates the valid time.
\item For forecast products, or the first of a series of forecast products, the reference time indicates the valid time of the analysis upon which the (first) forecast is based.
\item Initialized analysis products are allocated code figures distinct from those allocated to uninitialized analysis products.
\item Code figure 10 allows the period of a forecast to be extended over two octets; this is to assist with extended range forecasts.
\item Where products or a series of products are averaged or accumulated, the number involved is to be represented in octets 22 and 23 of Section 1, while any number missing is to be represented in octet 24.
\item Forecasts of the accumulation or difference of some quantity (e.g. quantitative precipitation forecasts), indicated by values of 4 or 5 in octet 21, have a product valid time given by the reference time + P2; the period of accumulation, or difference, can be calculated as $P2 - P1$.
\item A few examples may help to clarify the use of Code table 5:
  \begin{itemize}
  \item For analysis products, P1 will be zero and the time range indicator
	(octet 21) will also be zero; for initialized products (sometimes
	called "zero hour forecasts"), P1 will be zero, but octet 21 will be
	set to 1.
  \item For forecasts, typically, P1  will contain the number of hours of the
	forecast (the unit indicator given in octet 18 would be 1) and octet 21
	would contain a zero.
  \end{itemize}
\item Code value 115 would be used, typically, for multiple day mean forecasts,
      all derived from the same initial conditions.
\item Code value 117 would be used, typically, for Monte Carlo type
      calculations; many forecasts valid at the same time from different
      initial (reference) times.
\item Averages, accumulations, and differences get a somewhat specialized
      treatment. If octet 21 has a value between 2 and 5 (inclusive), then the
      reference time + P1  is the initial date/time and the reference time + P2
      is the final date/time of the period over which averaging or accumulation
      takes place. If, however, octet 21 has a value of 113, 114, 115, 116,
      117, 123, or 124, the P2 specifies the time interval between each of the
      fields (or the forecast initial times) that have been averaged or
      accumulated. These latter values of octet 21 require the quantities
      averaged to be equally separated in time; the former values, 3 and 4 in
      particular, allow for irregular or unspecified intervals of time between
      the fields that are averaged or accumulated.
\end{itemize}


\chapter{Contents of the local B table}

\section{B table contents}

\label{btable}
\input{../tables/dballe.tex}

\section{Extra code tables}

Code table 33196 \emph{Data has been invalidated}:

\begin{tabular}{|l|l|}
\hline
{\em Value} & {\em Description}               \\
\hline
          0 & data has been invalidated       \\
\hline
\end{tabular}

When the attribute B33196 is not present, then the data has not been
invalidated.

Code table 33197 \emph{Manual replacement in substitution}:

\begin{tabular}{|l|l|}
\hline
{\em Value} & {\em Description}               \\
\hline
          0 & a substitution happened, and this is the original value    \\
          1 & a substitution happened, and this is the substituted value \\
\hline
\end{tabular}

When the attribute B33197 is not present, then no substitution happened.


\chapter{Variable aliases}

\label{aliastable}
\input{../dballe/core/aliases.tex}


\chapter{Summary of routines}

This appendix contains some reference tables for the functions in the Fortran
API.

\section{Input/Output routines}

These routines are used to set the input and read the output of action routines.

\begin{tabular}{|l|l|l|}
\hline
{\em Name}                      & {\em Description}                & {\em Documented in} \\
\hline
{\tt idba\_seti(h, param, val)} & Set an integer in input          & \ref{fun-idba_set} \\
{\tt idba\_setr(h, param, val)} & Set a real in input              & \ref{fun-idba_set} \\
{\tt idba\_setd(h, param, val)} & Set a double in input            & \ref{fun-idba_set} \\
{\tt idba\_setc(h, param, val)} & Set a character in input         & \ref{fun-idba_set} \\
{\tt idba\_unset(h, param)}     & Unset a value from the input     & \ref{fun-idba_unset} \\
{\tt idba\_unsetall(h)}         & Unset all values from the input  & \ref{fun-idba_unsetall} \\
{\tt idba\_enqi(h, param, val)} & Read an integer from the output  & \ref{fun-idba_enq} \\
{\tt idba\_enqr(h, param, val)} & Read a real from the output      & \ref{fun-idba_enq} \\
{\tt idba\_enqd(h, param, val)} & Read a double from the output    & \ref{fun-idba_enq} \\
{\tt idba\_enqc(h, param, val)} & Read a character from the output & \ref{fun-idba_enq} \\
\hline
\end{tabular}


\section{Session routines}

These routines are used to begin and end working sessions with \dballe{}.

\begin{tabular}{|l|l|l|}
\hline
{\em Name} & {\em Description} & {\em Documented in} \\
\hline
{\tt idba\_presentati(dbh, dsn, user, password)} & Connect to the database & \ref{fun-idba_presentati} \\
{\tt idba\_preparati(dbh, h, anamode, datamode, attrmode)} & Open a new session & \ref{fun-idba_preparati} \\
{\tt idba\_fatto(h)}                        & Close a session              & \ref{fun-idba_fatto} \\
{\tt idba\_arrivederci(dbh)}                & Disconnect from the database & \ref{fun-idba_arrivederci} \\
\hline
\end{tabular}


\section{Action routines}

\begin{tabular}{|l|l|l|}
\hline
{\em Name} & {\em Description} & {\em Documented in} \\
\hline
{\tt idba\_scopa(h, filename)}     & Reinitialise the database       & \ref{fun-idba_scopa} \\
{\tt idba\_quantesono(h, count)}   & Query the stations archive      & \ref{fun-idba_quantesono} \\
{\tt idba\_elencamele(h)}          & Fetch data about a station      & \ref{fun-idba_elencamele} \\
{\tt idba\_voglioquesto(h, count)} & Query the data archive          & \ref{fun-idba_voglioquesto} \\
{\tt idba\_dammelo(h, param)}      & Fetch one observed data         & \ref{fun-idba_dammelo} \\
{\tt idba\_prendilo(h)}            & Insert values in the database   & \ref{fun-idba_prendilo} \\
{\tt idba\_dimenticami(h)}         & Remove values from the database & \ref{fun-idba_dimenticami} \\
{\tt idba\_voglioancora(h, count)} & Query QC data about a value     & \ref{fun-idba_voglioancora} \\
{\tt idba\_ancora(h, param)}       & Fetch QC data                   & \ref{fun-idba_ancora} \\
{\tt idba\_critica(h)}             & Insert QC data about a value    & \ref{fun-idba_critica} \\
{\tt idba\_scusa(h)}               & Remove QC data                  & \ref{fun-idba_scusa} \\
\hline
\end{tabular}

\section{Error management routines}

\begin{tabular}{|l|l|l|}
\hline
{\em Name} & {\em Description} & {\em Documented in} \\
\hline
{\tt idba\_error\_code()}              &
	Return the error code of the last function &
	\ref{fun-idba_error_code} \\
{\tt idba\_error\_message(msg)}        &
	Get the error message of the last function &
	\ref{fun-idba_error_message} \\
{\tt idba\_error\_context(msg)}        &
	Get the context description for the error of the last function &
	\ref{fun-idba_error_context} \\
{\tt idba\_error\_details(msg)}        & 
	Get details about the error of the last function &
	\ref{fun-idba_error_details} \\
{\tt idba\_error\_set\_callback(code, fun, data, h)} &
	Set a callback for handling errors &
	\ref{fun-idba_error_set_callback} \\
{\tt idba\_error\_remove\_callback(h)} &
	Remove a previously set callback &
	\ref{fun-idba_error_remove_callback} \\

{\tt idba\_scopa(h, filename)}     & Reinitialise the database       & \ref{fun-idba_scopa} \\
{\tt idba\_quantesono(h, count)}   & Query the stations archive      & \ref{fun-idba_quantesono} \\
{\tt idba\_elencamele(h)}          & Fetch data about a station      & \ref{fun-idba_elencamele} \\
{\tt idba\_voglioquesto(h, count)} & Query the data archive          & \ref{fun-idba_voglioquesto} \\
{\tt idba\_dammelo(h, param)}      & Fetch one observed data         & \ref{fun-idba_dammelo} \\
{\tt idba\_prendilo(h)}            & Insert values in the database   & \ref{fun-idba_prendilo} \\
{\tt idba\_dimenticami(h)}         & Remove values from the database & \ref{fun-idba_dimenticami} \\
{\tt idba\_voglioancora(h, count)} & Query QC data about a value     & \ref{fun-idba_voglioancora} \\
{\tt idba\_ancora(h, param)}       & Fetch QC data                   & \ref{fun-idba_ancora} \\
{\tt idba\_critica(h)}             & Insert QC data about a value    & \ref{fun-idba_critica} \\
{\tt idba\_scusa(h)}               & Remove QC data                  & \ref{fun-idba_scusa} \\
\hline
\end{tabular}


\bibliography{local}

\end{document}


